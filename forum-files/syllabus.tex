\documentclass[11pt, oneside]{article}   
\usepackage{geometry}   
\geometry{letterpaper}   
\usepackage[parfill]{parskip}
\usepackage{graphicx,url}			
\usepackage{amssymb}
\usepackage{multicol}
\usepackage{hyperref}
\hypersetup{
    colorlinks=true,
    linkcolor=blue,
    filecolor=magenta,      
    urlcolor=blue,
}
\setcounter{tocdepth}{1}

\title{Applied Category Theory School 2019}
\author{}
\date{}

\begin{document}

\maketitle
\tableofcontents

\pagebreak

%============================
\section{Overview}
\label{sec:overview}
% ============================

Each group is responsible for the following output:
\begin{itemize}
\item Two presentations (see Section \ref{sec:presentations})
\item Two articles (see Section \ref{sec:blog})
\end{itemize}
Each group can determine how to divide up the work, but we
encourage collaboration over a disjoint work style.  

In addition, each individual participant is responsible for
attending every presentation and posting a reading response
for each paper. See Section \ref{sec:responses} for details.

\pagebreak

% =====================================
\section{Project Descriptions}
\label{sec:projects}
% =====================================

%
% PROJECT 1
%

\subsection*{Group 1}

\textsc{Title.}
Simplifying Quantum Circuits using the ZX-Calculus

\textsc{Members.}
\begin{multicols}{2}
  \begin{itemize}
  \item Fatima Ahmadi (paper \ref{mb2})
  \item Miriam Backens (project lead)
  \item Hector Bakewell (teaching assistant)
  \item Giovanni de Felice (paper \ref{mb1})
  \item Leo Lobski (paper \ref{mb1})
  \item John van de Wetering (paper \ref{mb2})
  \end{itemize}
\end{multicols}

\textsc{Description.}  The ZX-calculus is a graphical
calculus based on the category-theoretical formulation of
quantum mechanics. A complete set of graphical rewrite rules
is known for the ZX-calculus, but not for quantum circuits
over any universal gate set. In this project, we aim to
develop new strategies for using the ZX-calculus to simplify
quantum circuits.

\textsc{Reading.}
\begin{enumerate}
\item \label{mb1}
  Matthew Amy, Jianxin Chen, Neil Ross. A finite
  presentation of CNOT-Dihedral operators. Available at
  \begin{quote}
    \href{https://arxiv.org/abs/1701.00140}{https://arxiv.org/abs/1701.00140}
  \end{quote}
\item \label{mb2}
  Miriam Backens. The ZX-calculus is complete for stabilizer
  quantum mechanics. Available at
  \begin{quote}
    \href{https://arxiv.org/abs/1307.7025}{https://arxiv.org/abs/1307.7025}
  \end{quote}
\end{enumerate}

\pagebreak

%
% PROJECT 2
%

\subsection*{Group 2}

\textsc{Title.}  Partial Evaluations, the Bar
Construction, and Second Order Stochastic Dominance

\textsc{Members.}
\begin{multicols}{2}
  \begin{itemize}
  \item Nathan Bedell (paper \ref{tf2})
  \item Carmen Constantin (paper \ref{tf2})
  \item Tobias Fritz (project lead)
  \item Martin Lundfall (paper \ref{tf1})
  \item Paolo Perrone (teaching assistant)
  \item Brandon Shapiro (paper \ref{tf1})
  \end{itemize}
\end{multicols}


\textsc{Description.}
We all know that 2+2+1+1 evaluates to 6. A less familiar
notion is that it can *partially evaluate* to 5+1. In this
project, we aim to study the compositional structure of
partial evaluation in terms of monads and the bar
construction and see what this has to do with financial risk
via second-order stochastic dominance.

\textsc{Reading.}
\begin{enumerate}
\item \label{tf1}
  Tobias Fritz, Paolo Perrone. Monads, partial evaluations,
  and rewriting. Available at
  \begin{quote}
    \href{https://arxiv.org/abs/1810.06037}{https://arxiv.org/abs/1810.06037}
  \end{quote}
\item \label{tf2}
  Maria Manuel Clementino, Dirk Hofmann, George
  Janelidze. The monads of classical algebra are seldom
  weakly Cartesian. Available at
  \begin{quote}
 \href{https://link.springer.com/article/10.1007/s40062-013-0063-2}{https://link.springer.com/article/10.1007/s40062-013-0063-2}
  \end{quote}
\item (extra)
  Todd Trimble. On the bar construction. Available at
  \begin{quote} \href{https://golem.ph.utexas.edu/category/2007/05/on\_the\_bar\_construction.html}{https://golem.ph.utexas.edu/category/2007/05/on\_the\_bar\_construction.html}
  \end{quote}
\end{enumerate}

\pagebreak

%
% PROJECT 3
%

\subsection*{Group 3}

\textsc{Title.}  Toward a Mathematical Foundation
for Autopoiesis

\textsc{Members.}
\begin{multicols}{2}
  \begin{itemize}
  \item Brendan Fong (teaching assistant)
  \item Bruno Gavranovic (paper \ref{ds1})
  \item Sophie Libkind (paper \ref{ds1})
  \item David Myers (paper \ref{ds2})
  \item Toby Smithe (paper \ref{ds2})
  \item David Spivak (project lead)
  \end{itemize}
\end{multicols}

\textsc{Description.}  An autopoietic organization---anything
from a living animal to a political party to a football
team---is a system that is responsible for adapting and
changing itself, so as to persist as events unfold. We want
to develop mathematical abstractions that are suitable to
found a scientific study of autopoietic organizations. To do
this, we'll begin by using behavioral mereology and
graphical logic to frame a discussion of autopoiesis, most
of all what it is and how it can be best conceived. We do
not expect to complete this ambitious objective; we hope
only to make progress toward it.

\textsc{Reading.}
\begin{enumerate}
\item \label{ds1}
  Fong, Myers, Spivak. Behavioral mereology. Available at
  \begin{quote}
    \href{https://arxiv.org/abs/1811.00420}{https://arxiv.org/abs/1811.00420}
  \end{quote}
\item \label{ds2}
  Fong, Spivak. Graphical regular logic. Available at
  \begin{quote}
    \href{https://arxiv.org/abs/1812.05765}{https://arxiv.org/abs/1812.05765}
  \end{quote}
\item (extra) Salge, Glackin, Polani. Changing the
  environment based on empowerment as intrinsic
  motivation. Available at
  \begin{quote}
    \href{https://www.mdpi.com/1099-4300/16/5/2789/htm}{https://www.mdpi.com/1099-4300/16/5/2789/htm}
  \end{quote}
\item (extra) Inevitable life (video). Available
  \begin{quote} \href{https://www.youtube.com/watch?v=ElMqwgkXguw\&feature=youtu.be\&t=1}{https://www.youtube.com/watch?v=ElMqwgkXguw\&feature=youtu.be\&t=1}
  \end{quote}
\item (extra)
  Luhmann. Organization and Decision, CUP. (Chapter 2)
\end{enumerate}

\pagebreak

%
% PROJECT 4
%

\subsection*{Group 4}

\textsc{Title.}  Formal and Experimental Methods to
Reason about Dialogue and Discourse using Categorical Models
of Vector Space

\textsc{Members.}
\begin{multicols}{2}
  \begin{itemize}
  \item Adriana Correia (paper \ref{ms1})
  \item Lachlan McPheat (paper \ref{ms1})
  \item Mehrnoosh Sadrzadeh (project lead)
  \item Dan Shiebler (paper \ref{ms2})
  \item Alexis Toumi (paper \ref{ms2})
  \item Gijs Wijnholds (teaching assistant)
  \end{itemize}
\end{multicols}


\textsc{Description.}
Distributional semantics argues that meanings of words can
be represented by the frequency of their co-occurrences in
context. A model extending distributional semantics from
words to sentences has a categorical interpretation via
Lambek's syntactic calculus or pregroups. In this project,
we intend to further extend this model to reason about
dialogue and discourse utterances where people interrupt
each other, there are references that need to be resolved,
disfluencies, pauses, and corrections. Additionally, we
would like to design experiments and run toy models to
verify predictions of the
developed models.

\textsc{Reading.}
\begin{enumerate}
\item \label{ms1}
  Gerhard Jager. A multi-modal analysis of anaphora and
  ellipsis. Available at
  \begin{quote}    \href{http://citeseerx.ist.psu.edu/viewdoc/summary?doi=10.1.1.53.7214}{http://citeseerx.ist.psu.edu/viewdoc/summary?doi=10.1.1.53.7214}
  \end{quote}
\item \label{ms2}
  Matthew Purver, Ronnie Cann, Ruth Kempson. Grammars as
  parsers: Meeting the dialogue challenge. Available
  \begin{quote} \href{http://www.eecs.qmul.ac.uk/~mpurver/papers/purver-et-al06rolc.pdf}{http://www.eecs.qmul.ac.uk/~mpurver/papers/purver-et-al06rolc.pdf}
  \end{quote}
\item (extra) Bob Coecke, Edward Grefenstette, Mehrnoosh
  Sadrzadeh. Lambek vs Lambek: Functorial vector space
  semantics and string diagrams for Lambek
  calculus. Available at
  \begin{quote} \href{https://www.sciencedirect.com/science/article/pii/S0168007213000626}{https://www.sciencedirect.com/science/article/pii/S0168007213000626}
  \end{quote}
\item (extra) Bob Coecke, Mehrnoosh Sadrzadeh, Stephen
  Clark. Mathematical foundations for a compositional
  distributional model of meaning. Available at
  \begin{quote}
    \href{https://arxiv.org/abs/1003.4394}{https://arxiv.org/abs/1003.4394}
  \end{quote}
\end{enumerate}

\pagebreak

%
% PROJECT 5
%

\subsection*{Group 5}

\textsc{Title.}  Complexity Classes, Computations,
and Turing Categories

\textsc{Members.}
\begin{multicols}{2}
  \begin{itemize}
  \item Georgios Bakirtzis (paper \ref{ph1})
  \item Adam Conghaile (paper \ref{ph2})
  \item Jonathan Gallagher (teaching assistant)
  \item Pieter Hofstra (project lead)
  \item Diego Roque (paper \ref{ph2})
  \item Christian Williams (paper \ref{ph1})
  \end{itemize}
\end{multicols}

\textsc{Description.}
Turing categories form a categorical setting for studying
computability without bias towards any particular model of
computation. It is not currently clear, however, that Turing
categories are useful to study practical aspects of
computation such as complexity. This project revolves around
the systematic study of step-based computation in the form
of stack-machines, the resulting Turing categories, and
complexity classes. This will involve a study of the
interplay between traced monoidal structure and
computation. We will explore the idea of stack machines qua
programming languages, investigate the expressive power, and
tie this to complexity theory. We will also consider
questions such as the following: can we characterize Turing
categories arising from stack machines? Is there an initial
such category? How does this structure relate to other
categorical structures associated with computability?

\textsc{Reading.}
\begin{enumerate}
\item \label{ph1}
  J.R.B. Cockett, P.J.W. Hofstra. Introduction to Turing
  categories. APAL, Vol 156, pp 183-209, 2008. Available at
  \begin{quote} \href{https://www.sciencedirect.com/science/article/pii/S0168007208000948}{https://www.sciencedirect.com/science/article/pii/S0168007208000948}
  \end{quote}
\item \label{ph2}
  J.R.B. Cockett, P.J.W. Hofstra, P. Hrubes. Total maps of
  Turing categories. ENTCS (Proc. of MFPS XXX), pp 129-146,
  2014. Available at
  \begin{quote} \href{https://www.sciencedirect.com/science/article/pii/S1571066114000759}{https://www.sciencedirect.com/science/article/pii/S1571066114000759}
  \end{quote}
\item (extra)
  A. Joyal, R. Street, D. Verity. Traced monoidal
  categories. Mat. Proc. Cam. Phil. Soc. 3, pp. 447-468,
  1996. Available at
  \begin{quote} \href{https://pdfs.semanticscholar.org/c232/37a187d026b8130d98c09187b8ba4f611c40.pdf}{https://pdfs.semanticscholar.org/c232/37a187d026b8130d98c09187b8ba4f611c40.pdf}
  \end{quote}
\end{enumerate}

\pagebreak

%
% PROJECT 6
%

\subsection*{Group 6}

\textsc{Title.}
Traversal Optics and Profunctors

\textsc{Members.}
\begin{multicols}{2}
  \begin{itemize}
  \item Bryce Clarke (paper \ref{bm1})
  \item Derek Elkins (teaching assistant)
  \item Mario Roman (paper \ref{bm2})
  \item Fosco Loregian (paper \ref{bm1})
  \item Bartosz Milewski (project lead)
  \item Emily Pillmore (paper \ref{bm2})
  \end{itemize}
\end{multicols}

\textsc{Description.}
In functional programming, optics are ways to zoom into a
specific part of a given data type and mutate it. Optics
come in many flavors such as lenses and prisms and there is
a well-studied categorical viewpoint, known as profunctor
optics. Of all the optic types, only the traversal has
resisted a derivation from first principles into a
profunctor description. This project aims to do just this.

\textsc{Reading.}
\begin{enumerate}
\item \label{bm1}
  Craig Pastro, Ross Street. Doubles for monoidal
  categories. Available at
  \begin{quote}
    \href{https://arxiv.org/abs/0711.1859}{https://arxiv.org/abs/0711.1859}
  \end{quote}
\item \label{bm2}
  Bartosz Milewski. Profunctor optics, categorical
  View. Available at
  \begin{quote} \href{https://bartoszmilewski.com/2017/07/07/profunctor-optics-the-categorical-view/}{https://bartoszmilewski.com/2017/07/07/profunctor-optics-the-categorical-view/}
  \end{quote}
\end{enumerate}

\pagebreak

% =======================================
\section{Schedule}
\label{sec:schedule}
% =======================================

The exact meeting times for the video conference will be
determined in the first week of the school. 

18-22 feb // introductions \\
25 feb-1 march //  a finite presentation \\
4-8 march // group 1 // the zx-calculus is complete \\
11-15 march // break \\
18-22 march // monads, partial evaluations, and rewriting \\
25-29 march // seldom weakly cartesian \\
1-5 april // break \\
8-12 april // behavioral mereology \\
15-19 april // graphical regular logic \\
22-26 april // break \\
29 april-3 may // anaphora and ellipsis \\
6-10 may // grammars as parsers \\
13-17 may // break \\
20-24 may  // turing categories \\
27-31 may  // total maps \\
3-7 june // break \\
10-14 june // doubles for monoidal categories \\
17-21 june// profunctor optics \\
24-28 june // break \\
8-12 july // ct2019 \\
15-19 july // act2019 conference \\
22-26 july // act2019 school \\

\pagebreak

% =======================================
\section{Presentations}
\label{sec:presentations}
% =======================================

Each group will make two presentations, one per paper.  For
those groups with additional readings, the content labeled
``extra'' is provided as an additional resource to
strengthen your background knowledge. Section
\ref{sec:projects} lists the paper you were assigned to
present and write about. You will have a partner.  Each
partnership may determine how to split the workload. We
encourage collaboration above a disjoint workflow.  

Due to presenting via video conference, using slides is
highly recommended. If this is not possible, a board talk
can work, but this can lead to difficulties in hearing the
speaker as they will be further from the microphone and
often turned to the board while writing.  

Each presentation should be 40 minutes, which will allow 20
minutes for conversation.  Please take care to adhere to
this timing as a courtesy for all participants.

Also, \textbf{please push any slides or notes} to the school
github repo prior to your presentation. Technical details
are in Section \ref{sec:github}.

\pagebreak

% =======================================
\section{Blog Posts}
\label{sec:blog}
% =======================================

Each group will write two articles, one per paper. This does
not need to be a summary of the paper. Indeed, try to
incorporate the project and related papers in this
school. Both pair of partners will receive authorship of the
paper, regardless of who does the majority writing.

We expect this article to be of professional quality. Thus,
it will go through several rounds of revisions.  After you
complete your first draft, allow your TA and project lead to
look at it.  Also, allow the organizers Daniel and Jules to
look.  Incorporate any notes you are given.

\textbf{Please complete your first draft by one week
  following your presentation.} This will allow enough time
for reviews and edits to be made, with the goal of posting
the article two weeks after you give your presentation.

Because it will be posted as a blog article, there is no
maximum length. However, we do want busy readers to make it
to the end of the article, so keep the readers attention
span in mind.

The article will be posted on the
\href{https://golem.ph.utexas.edu/category/}{nCafe}. To do so successfully requires some technical
knowledge which we discuss below.

% ~~~~~~~~~~~~~~~~~~~~~~~~~~~~~~~~~~~~~~~
\subsection{Posting to the nCat Cafe}
\label{sec:ncat}
% ~~~~~~~~~~~~~~~~~~~~~~~~~~~~~~~~~~~~~~~

For formatting, the nCafe uses `Markdown with MathML and
iTeX'. It's a basic mark-up language, based on Markdown and
TeX.  You can find some examples of it in the blog posts
folder of this github repo, and that should demonstrate
everything you need.

\textbf{Before posting, it is important that you test your
  entry.} To do this, go to the nCafe, choose any entry. Go
to the comment section as if you will comment on the
article. Copy and paste your entry into the comment box and
press preview, you'll see how it renders. (Except for
headings, ie except for the \#\# syntax. Commenters aren't
allowed headings.)


A few points to note about blog posts:
\begin{itemize}
  \item It's best if the first line of the text file is the
    title (short and snappy), and the second has this form:
    \begin{quote}
      *guest post by [author1](author1's url) and
      [author2](author2's website)
    \end{quote}
    This second line provides a clickable link your your
    website.

  \item There needs to be a clean break between the stuff
    that shows up on the blog's main page---which should be
    short, about 1 paragraph---and the rest of the entry.
    The first part is an advertisement for the second.

  \item Due to a defect of the nCafe, letters in math
    mode that appear without spaces between them, like
    \texttt{c\_\{ij\}}, will not be italicized.

  \item Subject headers starting with \#\# show up bigger
    than subject headers starting with \#\#\#, etc. (But
    again, you won't see this in comment preview.)
  \item The blog takes ASCII code and it runs on HTML, so
    special characters needed to be replaced by their HTML
    code. For example, \"a should be replaced by
    \texttt{\&auml;}, \'e by \texttt{\&eacute;}, $<$ by
    \texttt{\&lt;}, and so on.
\end{itemize}

Including images is encouraged, but needs a bit of
care. First, \emph{images must be .pngs with transparent
  background} (not .jpgs with a white background), else they
stand out on the gray background of the $n$-Caf\'e.
\footnote{
  For comparison of different image formats,
  pretend to post this comment on the n-Cafe, and preview
  it:
  \begin{quote}
    Nontransparent:
    \texttt{<img src = "http://math.ucr.edu/home/baez/mathematical/ACT2018/cicala/ diag\_snake-eq\_nontransparent.png"
      alt = ""/>}
    
    versus transparent:

    \texttt{
      <img src =
      "http://math.ucr.edu/home/baez/mathematical/ACT2018/cicala/ diag\_snake-eq.png"
      alt = ""/>}

    versus transparent and 300 pixels wide:

    \texttt{
      <img width = "300" src =
      "http://math.ucr.edu/home/baez/mathematical/ACT2018/cicala/diag\_snake-eq.png"
      alt = ""/>)}
  \end{quote}
}

Once in the correct format, please push your images to
\begin{quote}
  \texttt{http://danielmichaelcicala.github.io/act2019/\emph{assets/image.type}}
\end{quote}
Contact Daniel if you are having difficulty with this.

To include these in your blog post, use the following code:
\begin{quote}
  \texttt{<img width = "100" src =
    "http://danielmichaelcicala.github.io/act2019/\emph{assets/image.type}"
    alt = ""/>}
\end{quote}
You specify the width (or height) in pixels.

% =======================================
\section{Reading Responses}
\label{sec:responses}
% =======================================

Reading responses are a critical part of this school. Every
student is responsible for writing a reading response for
every paper other than that which they were assigned.  Some
prompts include
\begin{multicols}{2}
  \begin{itemize}
  \item questions you have
  \item ideas you have
  \item impressions made
  \item related facts
  \item what you liked or disliked
  \item speculating about future work
  \end{itemize}
\end{multicols}

These need not be long. The goal is to show that you engaged
with the material. If you struggled with the paper, say so,
ask questions, and others will help.  If the paper was an
easy read for you, answer questions and help clarify
concepts for others.  

\textbf{The reading responses are due the day prior to the associated
presentation.} They will be posted to the github repo (see
Section \ref{sec:github}). 

Specifically, all responses will be written in a single .tex
file named ``forum.tex''. We will start with a basic preamble, but feel free to
add any macro's or packages that will allow you to post more
comfortably.

We also encourage conversation on the forum.  To do
encourage this and also uniform formatting, there are two
macros you should use.
\begin{itemize}
\item \texttt{\textbackslash iam\{NAME\}} where you fill in
  your own name in place of NAME.  You write your primary
  reading response following it.
\item \texttt{\textbackslash respond\{NAME\}} is used to
  respond to another participants reading response.  This is
  used to answer questions, ask questions, provide
  references, etc.
\end{itemize}
More details will be contained in the forum.tex file.  

\pagebreak

% =======================================
\section{Video Conferencing}
\label{sec:video}
% =======================================

We will use the \href{https://zoom.us/}{zoom} video
conferencing software. It is free to sign-up and you can use
it in your browser or download an app for desktop or smartphone.

Go to the following url to join the meeting.

\begin{quote}
  https://ucrengage.zoom.us/j/4336013223
\end{quote}

During presentations, we ask that you keep your microphone
muted while you are not speaking. 

\pagebreak

% =======================================
\section{GitHub}
\label{sec:github}
% =======================================

Our forum is hosted on a shared LaTeX file on GitHub.
GitHub is an online service that hosts Git repositories
(repos for short). Git is a software package for sharing and
simultaneously editing documents. It was developed to
organize large open source coding projects, like writing
parts of the Linux operating system. It inspired Dropbox,
and indeed you can think about it as a more advanced form of
Dropbox.

We considered using Dropbox for sharing a LaTeX file, but
the problem with that is that Dropbox doesn't know what to
do if two people edit the file simultaneously. This can lead
to people losing work, and is rather frustrating.

Git solves this problem by making everyone have their own
copies, and then making everyone run explicit commands to
merge and update their own copies with a central master
copy.

Our forum is private, so you'll need to have a GitHub
account. Sign up at
\begin{quote}
  \url{https://github.com/join?source=header-home}
\end{quote}
Once you have created your account, please email your
username to Daniel, so he can give you access.  Once you
have an account, download Git by clicking the relevant one
of these:
\begin{itemize}
\item  Windows \url{http://gitforwindows.org}
\item  Mac  \url{https://gist.github.com/derhuerst/1b15ff4652a867391f03#file-mac-md}
\item Linux  \url{https://gist.github.com/derhuerst/1b15ff4652a867391f03#file-linux-md}
\end{itemize}

How you interact with Git/GitHub will depend on your own
computer and whether you decide to use a GUI (e.g./ GitHub
Desktop)) or the terminal. 

Once your GitHub account is active and you've downloaded
git, you will want to clone the repo; this copies the files
contained in the repo to your computer. For this, you must
have already accepted our invitation to the private repo.

If you have a GUI, click clone and enter the URL
\begin{quote}
  \url{https://github.com/danielmichaelcicala/act2019}
\end{quote}
when asked.

If you are using terminal, navigate to the folder where you
want the forum to be created, and then run:

\begin{verbatim}
    git clone https://github.com/danielmichaelcicala/act2019
\end{verbatim}

Now you have the forum! It will be in a folder called
``act2019''.

All that remains is to use it. If you use a GUI, then using
Git should just be a matter of pressing the buttons they
tell you to.

If using terminal these are the steps you should follow to
post your reading response.

\begin{enumerate}
\item Open terminal and navigate to the
``act2019'' folder.

\item At the prompt, type ``\verb|git pull|''. This updates your
  local copy of the forum, so you can read the latest posts.

\item Open forum.tex and type your reading response, save
  the file, and generate the pdf to ensure there are no errors.
  
\item In terminal, type ``\verb|git add .|'' to add all of
  the updated files in the local copy of your
  ``act2019'' folder into a staging area where git has
  access to them.
  
\item In terminal, type ``\verb|git commit -m  "MESSAGE"|''. This tells git to save the added files and
  annotates them with the message you enter at MESSAGE. It's
  best to add a short ($ \sim $3 word) description of your
  changes. Something like ``Pam's reading response''.

\item At the terminal, type ``\verb|git push|''. This will merge
  your changes with the master copy and your response will
  be reflected in subsequent 'pulls'. 
\end{enumerate}





\end{document}