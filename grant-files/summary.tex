\documentclass[12pt]{amsart}

\sloppy

\addtolength{\textwidth}{60pt}
\addtolength{\oddsidemargin}{-32pt}
\evensidemargin\oddsidemargin

\setlength{\parskip}{1ex plus 0.5ex minus 0.2ex}

%PACKAGES
\usepackage{geometry}               
\geometry{letterpaper}                   

%URL colors
\usepackage{color}
\definecolor{myurlcolor}{rgb}{0.6,0,0}
\definecolor{mycitecolor}{rgb}{0,0,0.8}
\definecolor{myrefcolor}{rgb}{0,0,0.8}
\usepackage[bookmarks=false]{hyperref}
\hypersetup{colorlinks,
  linkcolor=myrefcolor,
  citecolor=mycitecolor,
  urlcolor=myurlcolor}

\newtheorem{thm}{Theorem}
\newtheorem{cor}[thm]{Corollary}
\newtheorem{fact}[thm]{Fact}
\newtheorem{lem}[thm]{Lemma}
\newtheorem{example}[thm]{Example}
\newtheorem{defn}[thm]{Definition}
\newtheorem{prop}[thm]{Proposition}
\newtheorem{sch}[thm]{Scholium}
\renewcommand{\AA}{\mathbf{A}}
\newcommand{\RR}{\mathbb{R}}
\newcommand{\QQ}{\mathbb{Q}}
\newcommand{\NN}{\mathbb{N}}
\newcommand{\CC}{\mathbb{C}}
\newcommand{\ZZ}{\mathbb{Z}}
\newcommand{\CM}{\mathrm{CM}}
\newcommand{\gen}{\mathrm{gen}}
\newcommand{\maps}{\colon}
\newcommand{\tensor}{\otimes}
\renewcommand{\thefootnote}{\fnsymbol{footnote}}
\newcommand {\tuple}[1]{\langle #1 \rangle}
\newcommand{\define}[1]{{\bf \boldmath{#1}}}


%%%%%%%%%% BEGIN DOCUMENT %%%%%%%%%%%%%%%%%%
\begin{document}

\centerline{\large PROJECT SUMMARY}
\title{Applied Category Theory 2019}
\author{John C.\ Baez}

\maketitle

\noindent {\bf Overview}.  Dr.\ Bob Coecke of Oxford University, has
agreed to host \textbf{Applied Category Theory 2019} in the summer of
2019. With funding from the NSF, grad students and postdocs from the US
would be able to attend this event which will bring together 30
senior researchers, 25 junior researchers, and 5 participants from
industry to discuss progress in applied category theory, create
community, and lay out a roadmap for future work.  From January to
April 2019, junior researchers will participate in an online reading
seminar, culminating in a research school in Oxford where they will
solve problems with help from expert mentors.  The workshop proper
will then feature lectures and discussion in areas throughout applied
category theory: chemical reaction networks, distributed systems,
information theory, functional programming, process calculii, chemical
systems, database theory, biological networks and diversity, and power
networks.  This event is organized by Bob Coecke (Oxford), Daniel
Cicala (UC Riverside), and Jules Hedges (Oxford).

\noindent \textbf{Intellectual merit.}  Category theory has long been a powerful tool for transferring techniques between disciplines in pure mathematics.  Recently, it has become a useful tool in disciplines ranging from biochemistry, electrical engineering and control theory to the study of stochastic processes, the design of databases, and the design of networks of mobile agents.   Operads, sheaves and other sophisticated concepts are not only giving rise to new methodologies for system design and analysis, but also triggering new developments in the underlying mathematics.

\noindent \textbf{Broader impacts.} 
Currently, category theory informs work at US companies including Metron Scientific Solutions (the PI is using operads in their project with DARPA to build a Complex Adaptive System Composition and Design Environment, Siemens (one student of the PI is working in their project Next-Generation Engineering with Category Theory and Sheaves), and Categorical Informatics (the one member of senior personnel for this proposal is a cofounder of this company, which uses category theory to design databases).   However, the potential of this subject in industry has just begun to be exploited.  This workshop will address this problem.  It will teach junior researchers in mathematics how to apply category theory, expose participants from industry to the uses of this subject, and forge new contacts between mathematicians and industry.  

\end{document}
