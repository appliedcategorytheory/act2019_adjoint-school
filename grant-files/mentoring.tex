\documentclass[12pt,letterpaper]{amsart}

\sloppy

\addtolength{\textwidth}{60pt}
\addtolength{\oddsidemargin}{-32pt}
\evensidemargin\oddsidemargin

\setlength{\parskip}{1ex plus 0.5ex minus 0.2ex}

%PACKAGES
\usepackage{geometry}               
\geometry{letterpaper}                   

%URL colors
\usepackage{color}
\definecolor{myurlcolor}{rgb}{0.6,0,0}
\definecolor{mycitecolor}{rgb}{0,0,0.8}
\definecolor{myrefcolor}{rgb}{0,0,0.8}
\usepackage[bookmarks=false]{hyperref}
\hypersetup{colorlinks,
  linkcolor=myrefcolor,
  citecolor=mycitecolor,
  urlcolor=myurlcolor}


%%%%%%%%%%%%%%  BEGIN DOCUMENT  %%%%%%%%%%%%%%%%%
\begin{document}

\centerline{\large MENTORING PLAN}
\title{Applied Category Theory 2019}
\author{John C.\ Baez}

\maketitle

The graduate students and post-doctorates supported by this proposal
will receive extensive mentoring from senior researchers (henceforth
called mentors):

\begin{enumerate}
\item From January to April 2019 there will be an online reading
  seminar attended by 15 graduate students, 5 post-doctorates (as
  TAs), 5 mentors, and 2 facilitators. The seminar will consist of 10 two-week
  blocks each devoted to one paper, chosen by a mentor, on applications of
  category theory as assigned reading and with three students as
  leaders. Each block will have three phases:
  \begin{itemize}
  \item a discussion of the assigned paper using an online
    videoconferencing system, consisting of a presentation by the
    student leaders followed by questions,
  \item responses on a private forum from the other participants,
    moderated by the facilitators and the associated TA serving as the
    domain expert,
  \item publication on the well-known mathematics blog \textsl{The
      $n$-Category Caf\'e} of an article authored by the student
    leaders that summarizes the assigned paper.
  \end{itemize}
  \vskip 1em
\item In the summer of 2019, there will be a research school in
  Oxford, UK attended by the 15 graduate students, 5 post-doctorates,
  5 mentors, and 2 organizers. After opening lectures by the mentors,
  time will be devoted to collaborative research.  Each mentor will
  work with their chosen post-doctorate and three students who led the
  online discussions on the two papers they assigned.  They will try
  to prove conjectures outlined by the mentors at the beginning of the
  seminar. The results will be written up for publication.  Lectures
  will be videotaped and placed on the \textsl{Oxford Quantum Video}
  website as well as the conference website.  \vskip 1em
\item In the week following the research school, the participants will
  attend a workshop with 60 participants, also in Oxford.  This focus
  will be on structured discussion, though there will also be three
  hours of invited talks per day.  They will join in discussions not
  only with 30 senior researchers in applied category theory, but also
  5 members of industry.  This will help them network and increase
  their ability to get jobs either in academia or in industry.
\end{enumerate}


%%%%%%%%%%%%%  END DOCUMENT  %%%%%%%%%%%%%%%%
\end{document}
