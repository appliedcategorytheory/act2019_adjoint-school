\documentclass[12pt,letterpaper]{amsart}

\sloppy

\addtolength{\textwidth}{60pt}
\addtolength{\oddsidemargin}{-32pt}
\evensidemargin\oddsidemargin

\setlength{\parskip}{1ex plus 0.5ex minus 0.2ex}

%PACKAGES
\usepackage{geometry}               
\geometry{letterpaper}                   

%URL colors
\usepackage{color}
\definecolor{myurlcolor}{rgb}{0.6,0,0}
\definecolor{mycitecolor}{rgb}{0,0,0.8}
\definecolor{myrefcolor}{rgb}{0,0,0.8}
\usepackage[bookmarks=false]{hyperref}
\hypersetup{colorlinks,
  linkcolor=myrefcolor,
  citecolor=mycitecolor,
  urlcolor=myurlcolor}

\begin{document}

\centerline{\large BIOGRAPHICAL SKETCH}
\vskip 1em
\centerline{David I. Spivak}

\vskip 1em
\vskip 1em

%---------------------------------
\textbf{Professional preparation}

\vskip 1em

\begin{itemize}
\item
  University of Maryland, College Park: \ \ BS, Mathematics, 1996
\item
  University of California, Berkeley: \ \ PhD, Mathematics, 2007 
\end{itemize}

%---------------------------------
\textbf{Appointments}

\vskip 1em

\begin{itemize}
\item
  2013 -- Present: Research Scientist, Massachusetts Institute of
  Technology
\item
  2010 -- 2013: Postdoctoral Associate, Massachusetts Institute of
  Technology
\item
  2007 -- 2010: Visiting Assistant Professor, University of Oregon
\end{itemize}

%------------------------------------
\textbf{Five products most closely related to the proposed project}

\vskip 1em

\begin{itemize}
\item
  \textbf{Spivak, D.I.} (2012) "Functorial Data Migration''.
  \emph{Information and Communication}. Vol 217, 31 -- 51.
\item
  \textbf{Spivak, D.I.} (2014) \emph{Category Theory for the
    Sciences}. Cambridge: MIT Press. 486 pages.
\item
  Giesa, T.; Jagadeesan, R.; \textbf{Spivak, D.I.}; Buehler,
  M.J. (2015) ``Matriarch: a Python library for materials
  architecture.'' \emph{ACS Biomaterials Science \&
    Engineering}, %\url{http://pubs.acs.org/doi/full/10.1021/acsbiomaterials.5b00251}.
\item
  \textbf{Spivak, D.I.}; Tan, J.Z. (2016) ``Nesting of dynamic
  systems and mode-dependent networks.'' \emph{Journal of Complex
    Networks.} %doi: 10.1093/comnet/cnw022.
\item
  Wisnesky, R.; Breiner, S.; Jones, A.; \textbf{Spivak, D.I.};
  Subrahmanian, E. (2017) ``Using category theory to facilitate
  multiple manufacturing service database integration.'' ASME. Journal
  of Computing and Information Science in Engineering 17(2), 021011.
\end{itemize}

%-----------------------------------------
\textbf{Five other significant products}

\vskip 1em

\begin{itemize}
\item
  Giesa, T.; \textbf{Spivak, D.I.}; Buehler, M.J. (2012) "Category
  theory based solution for the building block replacement problem in
  materials design''. \emph{Advanced Engineering Materials}. DOI:
  10.1002/adem.201200109
\item
  \textbf{Spivak, D.I.}; Kent, R.E. (2012) "Ologs: a categorical
  framework for knowledge representation''.  \emph{PLoS ONE} 7(1):
  e24274. doi:10.1371/journal.pone.0024274.
\item
  Gross, J.; Chlipala, A.; \textbf{Spivak, D.I.} (2014)
  "Experience Implementing a Performant Category-Theory Library in
  Coq''. \emph{5th conference on interactive theorem proving
    (ITP'14)}.
\item
  Vagner, D.; \textbf{Spivak, D.I.}; Lerman, E. (2015) ``Algebras
  of open dynamical systems on the operad of wiring diagrams.''
  \emph{Theory and Application of Categories} Vol.\ 30, No.\ 51,
  1793--1822.
\item
  \textbf{Spivak, D.I.} (2017) ``Categories as mathematical
  models.'' To appear in \emph{Categories for the Working
    Philosopher}. Oxford University Press.
\end{itemize}

%--------------------------------------
\textbf{Synergistic activities}

\vskip 1em

\begin{itemize}
\item
  I taught "Category theory for scientists" in Spring 2013 at MIT,
  a first-of-its-kind course on applied category theory. The 18
  students were from math, materials science, computer science,
  neuroscience, and other fields. The course textbook \emph{Category
    Theory for the Sciences}, published by MIT Press, has led to
  interest from people of a wide variety of backgrounds, both
  geographically and in terms of mathematical sophistication.
\item
  I hired and worked with Ryan Wisnesky to create a software tool,
  \href{http://categoricaldata.net/fql.html}{FQL}, which can be used
  to teach category theory, including left and right Kan extensions as
  "data migration functors". A company, \emph{Categorical Informatics
    Inc.}---partially supported by NSF I-Corps---has now been spun out
  of MIT to commercialize this product.
\item
  I have hired researchers from a variety of background as
  postdocs. Out of six postdocs hired to date, one is a woman, one is
  latino, and another is half latino.
\item
  Since 2010, I have mentored 21 undergraduate students of all
  backgrounds on research projects in applied category theory, for a
  total of 29 semesters (some students working with me more than
  once). Several of these projects resulted in published papers. I
  have also volunteered to tech two reading courses on category
  theory.
\item
  I co-organized the first Computational Category Theory
  conference at NIST in 2015 and the "Workshop on topology and
  abstract algebra for biomedicine" at the Pacific Symposium on
  Biocomputing (PSB) 2016, and the STRING 2017 conference in Oxford.
\end{itemize}

\end{document}

% \section*{Collaborations and other affiliations}

% \subsection{Collaborators and Co-Editors (20)}

% Spencer Breiner (NIST),
% Markus Buehler (MIT), 
% Andrea Censi (ETH),  
% Adam Chlipala (MIT),
% Subrahmanian Eswaran (CMU),
% Brendan Fong (MIT),   
% Henrik Forssell (University of Oslo),
% Tristan Giesa (MIT), 
% Jason Gross (MIT),
% Hakon R. Gylterud (Stockholm University),
% Al Jones (NIST),
% Eugene Lerman (UIUC),
% Marco P\'erez (University of Mexico),
% Dylan Rupel (Notre Dame),
% Patrick Schultz (MIT),
% Joshua Tan (Oxford),      
% Michael Triantafyllou (MIT), 
% R\'{e}my Tuy\'{e}ras (MIT), 
% Dmitry Vagner (Duke),
% Ryan Wisnesky (Categorical Informatics Inc).

% \subsection{Graduate Advisors and Postdoctoral Sponsors (4)}
% Tom Graber (Cal Tech), Jacob Lurie (Harvard), Daniel Dugger (University of Oregon), Haynes Miller (MIT).

% \subsection{Thesis Advisor (1)} 
% Peter Teichner (University of California, Berkeley).



% \end{document}

% \begin{center}\Large David I. Spivak\end{center}
% \noindent
% Department of Mathematics\\
% Massachusetts Institute of Technology\\
% 77 Massachusetts Avenue\\
% Building 2, Room 180\\
% Cambridge, MA 02139\\
% dspivak@mit.edu\\