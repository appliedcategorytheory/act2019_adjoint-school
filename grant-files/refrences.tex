\documentclass[12pt]{amsart}

\sloppy

\addtolength{\textwidth}{60pt}
\addtolength{\oddsidemargin}{-32pt}
\evensidemargin\oddsidemargin

\setlength{\parskip}{1ex plus 0.5ex minus 0.2ex}

%PACKAGES
\usepackage{geometry}               
\geometry{letterpaper}                   

%URL colors
\usepackage{color}
\definecolor{myurlcolor}{rgb}{0.6,0,0}
\definecolor{mycitecolor}{rgb}{0,0,0.8}
\definecolor{myrefcolor}{rgb}{0,0,0.8}
\usepackage[bookmarks=false]{hyperref}
\hypersetup{colorlinks,
  linkcolor=myrefcolor,
  citecolor=mycitecolor,
  urlcolor=myurlcolor}

%%%%%%%%%%%%%%%%%%  BEGIN DOCUMENT  %%%%%%%%%%%%%%%%%
\begin{document}

\centerline{\large REFERENCES}
\title{Applied Category Theory 2019}
\author{John C.\ Baez}

\maketitle

%%%%%%%%%%%%%%%%%%%%%%  BIB  %%%%%%%%%%%%%%%%%%%%%%%%
\begin{thebibliography}{10}

\bibitem{abramsky09} S.\ Abramsky and B.\ Coecke, A categorical semantics of quantum protocols, in \textsl{Handbook of Quantum Logic and Quantum Structures}, Elsevier, Amsterdam, 2009.  Also available at \href{https://arxiv.org/abs/quant-ph/0402130}{arXiv:quant-ph/0402130}.

\bibitem{dagstuhl14} S.\ Abramsky, J.\ C.\ Baez, F.\ Gadducci and V.\ Winschel, \textsl{Categorical Methods at the Crossroads}, Report from Dagstuhl Perspectives Workshop \textbf{14182}, 2014.  Available at \href{http://drops.dagstuhl.de/opus/volltexte/2014/4618/}{http://drops.dagstuhl.de/opus/volltexte/2014/4618/}.

\bibitem{albasini} L.\ de Francesco Albasini, N.\ Sabadini and Robert F.\ C.\ Walters, The compositional construction of Markov processes, \textsl{Appl.\ Cat.\ Str.\ }{\bf 19} (2011), 425--437.
Also available as \href{http://arxiv.org/abs/0901.2434}{arXiv:0901.2434}.

\bibitem{arbib05} M.\ A.\ Arbib and E.\ G.\ Manes. A categorist’s view of automata and systems, in \textsl{Category Theory Applied to Computation and Control}, E.\ G.\ Manes (ed.), Springer, Berlin, 2005.

\bibitem{baez15}  J.\ C.\ Baez and B.\ Fong. A compositional framework for passive linear networks.  Available at \href{https://arxiv.org/abs/1504.05625}{arXiv:1504.05625}.

\bibitem{baezfongpollard}  J.\ C.\ Baez, B.\ Fong and B.\ Pollard, \textsl{A compositional framework for Markov processes}, \textsl{Jour\. Math.\ Phys.} \textbf{57} (2016), 033301. Also available as \href{https://arxiv.org/abs/1508.06448}{arXiv:1508.06448}.

\bibitem{leinster11}  J.\ C.\ Baez, T.\ Fritz and T.\ Leinster, A characterization of entropy in terms of information loss, \textsl{Entropy} \textbf{13} (2011), 1945--1957.  Also available as \href{https://arxiv.org/abs/1106.1791}{arXiv:1106.179}.

\bibitem{baezpollard} J.\ C.\ Baez and B.\ Pollard, A compositional framework for reaction networks, 
\textsl{Rev.\ Math.\ Phys.} \textbf{29}, 1750028.  Also available as \href{https://arxiv.org/abs/1704.02051}{arXiv:1704.02051}.

\bibitem{baez11}  J.\ C.\ Baez and M.\ Stay, Physics, topology, logic and computation: a Rosetta Stone, in \textsl{New Structures for Physics}, ed.\ Bob Coecke, Springer, Berlin, 2011.  Also available as \href{https://arxiv.org/abs/0903.0340}{arXiv:0903.0340}.

\bibitem{erbele} J.\ C.\ Baez and J.\ Erbele, Categories in control,  \textsl{Th.\ Appl.\ Cat.\ }\textbf{30} (2015), 836--881.   Also available as
\href{http://arxiv.org/abs/1405.6881}{arXiv:1405.6881}.

\bibitem{bonchi} F.\ Bonchi, P.\ Soboci\'nski and F.\ Zanasi, A categorical semantics of signal flow graphs, in \textsl{CONCUR 2014--Concurrency Theory}, eds.\ P.\ Baldan and D.\ Gorla, \textsl{Lecture Notes in Computer Science} vol.\ 8704, Springer, Berlin, 2014, pp.\ 435--450.  Also available at \href{https://pdfs.semanticscholar.org/c908/47f1d138c9b44aaed72bcd59c9ec1915d395.pdf}{https://pdfs.semanticscholar.org/c908/47f1d138c9b44aaed72bcd59c9ec1915d395.pdf}.

\bibitem{spivak17} S.\ Breiner, A.\ Jones, D.\ Spivak, E.\ Subrahmanian and R.\ Wisnesky, Using category theory to facilitate multiple manufacturing service database integration, \textsl{ASME Journal of Computing and Information Science in Engineering} \textbf{17} (2017), 021011.  Available at \href{http://computingengineering.asmedigitalcollection.asme.org/article.aspx?articleid=2539429}{http://computingengineering.asmedigitalcollection.asme.org/article.aspx?}\break \href{http://computingengineering.asmedigitalcollection.asme.org/article.aspx?articleid=2539429}{articleid=2539429}.

\bibitem{cobbold} C.\ Cobbold and T.\ Leinster, Measuring diversity: the importance of species similarity, \textsl{Ecology} \textbf{93} (2012), 477--489.  Also available at \href{http://www.maths.ed.ac.uk/~tl/mdiss.pdf}{http://www.maths.ed.ac.uk/$\sim$tl/mdiss.pdf}

\bibitem{crole} R.\ Crole, \textsl{Categories for Types}, Cambridge U.\ Press, Cambridge, 1994.

\bibitem{freedman} M.\ Freedman, A.\ Kitaev, M.\ Larsen and Z.\ Wang, Topological quantum computation, \textsl{Bulletin of the AMS} \textbf{2003} \textbf{40}, 31–38.  Also available at \href{https://arxiv.org/abs/quant-ph/0101025}{arXiv:quant-ph/0101025}.

\bibitem{ghica}  D.\ R.\ Ghica and A.\ Jung, Categorical semantics of digital circuits, in \textsl{Proceedings of the 16th Conference on Formal Methods in Computer-Aided Design}, 
R.\ Piskac and M.\ Talupur (eds.), Springer, Berlin, 2016.  Also available at \href{https://www.cs.bham.ac.uk/~drg/papers/fmcad16.pdf}{https://www.cs.bham.ac.uk/$\sim$drg/papers/fmcad16.pdf}.

\bibitem{sadrzadeh}  D.\ Kartsaklis, M.\ Sadrzadeh, S.\ Pulman and B.\ Coecke, Reasoning about meaning in natural language with compact closed categories and Frobenius algebras, in \textsl{Logic and Algebraic Structures in Quantum Computing and Information}, Cambridge U.\ Press, Cambridge, 2013.   Also available as \href{https://arxiv.org/abs/1401.5980}{arXiv:1401.5980}.

\bibitem{leinster} T.\ Leinster, The magnitude of metric spaces, \textsl{Doc.\ Mathematica} \textbf{18} (2013), 857--905.   Also available as \href{https://arxiv.org/abs/1012.5857}{arXiv:1012.5857}.

\bibitem{grefenstette} E.\ Grefenstette and M.\ Sadrzadeh, Experimental support for a categorical compositional distributional model of meaning, \textsl{Proceedings of the Conference on Empirical Methods in Natural Language Processing}, Association for Computational Linguistics, 2011, 1394--1404.   Also available as \href{https://arxiv.org/abs/1106.4058}{arXiv:1106.4058}.

\bibitem{pierce} B.\ C.\ Pierce, \textsl{Basic Category Theory for Computer Scientists}, MIT Press, Cambridge Massachusetts, 1991.

\bibitem{moggi}  E.\ Moggi, Notions of computation and monads, \textsl{Information and Computation} \textbf{93} (1991), 55--92.

\bibitem{plotkin} G.\ Plotkin, A calculus of chemical systems, in \textsl{In Search of Elegance in the Theory and Practice of Computation: Essays Dedicated to Peter Buneman} eds.\ V.\ Tannen, 
\textit{et al}, Springer, Berlin, 2013, pp.\ 445--465.  Also available at \href{http://homepages.inf.ed.ac.uk/gdp/publications/CCS.pdf}{http://homepages.inf.ed.ac.uk/gdp/publications/CCS.pdf}.

\bibitem{Kan} E.\ Riehl, The Kan Extension Seminar: an experimental online graduate reading course, \textsl{AMS Notices} \textbf{61} (2014), 1357--1358.  Also available at \href{http://www.ams.org/notices/201411/rnoti-p1357.pdf}{http://www.ams.org/notices/201411/rnoti-p1357.pdf}.

\bibitem{riehl} E.\ Riehl, \textsl{Categories in Context},  Dover, New York, 2016.  Also available at \href{http://www.math.jhu.edu/~eriehl/context.pdf}{http://www.math.jhu.edu/~eriehl/context.pdf}.

\bibitem{selinger} P.\ Selinger and B.\ Valiron, Quantum lambda calculus, 
in Simon Gay and Ian Mackie, eds., \textsl{Semantic Techniques in Quantum Computation}, Cambridge U.\ Press, Cambridge, 2009, pp.\ 135--172.  Also available at \href{https://arxiv.org/abs/quant-ph/0307150}{arXiv:quant-ph/0307150}.

\bibitem{spivak12a} D.\ I.\ Spivak, Functorial data migration, \textsl{Information and Communication} \textbf{217} (2012), 31--51.   Also available as \href{https://arxiv.org/abs/1009.1166}{arXiv:1009.1166}.

\bibitem{spivak12b}  D.\ I.\ Spivak and R.\ E.\ Kent, Ologs: a categorical framework for knowledge representation, \textsl{PLoS ONE} (2012), e24274.    Also available as 
\href{https://arxiv.org/abs/1102.1889}{arXiv:1102.1889}.

\bibitem{spivak14}  D.\ I.\ Spivak, \textsl{Category Theory for Scientists}, MIT Press, Cambridge Massachusetts, 2014.

\bibitem{spivak16} D.\ I.\ Spivak, C.\ Vasilakopoulou and P.\ Schultz, Dynamical systems and sheaves.  Available at \href{https://arxiv.org/abs/1609.08086}{arXiv:1609.08086}.

\bibitem{vagner} D.\ Vagner, D.\ I.\ Spivak and E.\ Lerman, Algebras of open dynamical systems on the operad of wiring diagrams, \textsl{Th.\ Appl.\ Cat.} \textbf{30} (2015), 1793--1822.   Also available at \href{https://arxiv.org/abs/1408.1598}{arXiv:1408.1598}.

\bibitem{spivak15}  R.\ Wisnesky, D.\ I.\ Spivak, P.\ Schultz and E.\ Subrahmanian, Functorial data migration: from theory to practice.  Also available as \href{https://arxiv.org/abs/1502.05947}{arXiv:1502.05947}.
\end{thebibliography}


%%%%%%%%%%%%%%  END DOCUMENT  %%%%%%%%%%%%%%%%%%%%
\end{document}



